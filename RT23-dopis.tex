% \input utf8-t1
\documentclass[12pt,a4paper]{extarticle}

\usepackage[czech]{babel}
\usepackage[utf8]{inputenc}
\usepackage{psvectorian}
\usepackage[T1]{fontenc}
\usepackage{fontspec}

\usepackage[paper=a4paper,top=2cm,left=2.5cm,right=2.5cm,bottom=1cm]{geometry}

\usepackage{aurical}
\usepackage{inslrmaj}
\usepackage{yfonts}
\usepackage{type1cm}
\usepackage{lettrine}
\usepackage{graphicx}
\usepackage{color}
\usepackage{transparent}

\setcounter{secnumdepth}{1}
\setlength{\parindent}{0pt}
\setlength{\parskip}{0.5\bigskipamount}

\newfontfamily{\xiberonnettt}{XiBeronne}
\DeclareTextFontCommand{\textxib}{\xiberonnettt}

\linespread{1.3}
\pagestyle{empty}

\setmainfont{Xenippa}
%\setmainfont{XiBeronne}
\input Zallman.fd
\input Rothdn.fd
\input EileenBl.fd
\input Typocaps.fd

\newcommand*\capitalInit{\usefont{U}{EileenBl}{xl}{n}}
\newcommand*\romanCaps{\usefont{U}{Typocaps}{xl}{n}}
\usepackage{background}
\usepackage{eso-pic,graphicx}
\backgroundsetup{
	scale=1,
	color=black,
	opacity=0.95,
	angle=180,
	contents={\includegraphics[width=\paperwidth,height=\paperheight]{background.jpg}}%
}

\definecolor{myGray}{HTML}{22222A}
\color{myGray}
\renewcommand*{\psvectorianDefaultColor}{myGray}%

% \renewcommand{\normalsize}{\fontsize{10}{12}\selectfont}

\newcommand{\romanYear}{\xiberonnettt M\kern-0.14em M\kern-0.14em X\kern-0.14em X\kern-0.14em I\kern-0.14em I\kern-0.14em I}

\renewcommand*\ttdefault{dayrom}

\begin{document}

\begin{center}
%\rput[r](-11pt,9pt){\psvectorian[color=black,height=1.1cm]{11}}%
%\rput[l](15pt,9pt){\psvectorian[color=black,height=1.1cm,mirror]{11}}%
{\textxib \bfseries \fontsize{32}{16}\selectfont Pod Praporem \romanYear}%
\end{center}\vspace{-12pt}
\begin{center}
  \psvectorian[width=.8\textwidth]{86}
\end{center}\vspace{0pt}

{\large Bratři a Sestry!}\vspace{-12pt}
\lettrine[lines=2,slope=0pt,findent=0pt,nindent=0pt,lraise=-0.2,loversize=0.5]{\capitalInit L}{idu} všemu, jinochům, pannám, zbrojným i lidu prostému, každému, kdos Ducha rytířství v sobě má či po něm prahne a chovati ho ve svém srdci chce. Zpozorni a list tento bedlivě čti!

\lettrine[lines=2,slope=0pt,findent=0pt,nindent=0pt,lraise=-0.2,loversize=0.5]{\capitalInit V}{elké} věci, jež v našem okolí se dějí, na každéhos dopadají. Člověka, jež by nevěděl v zemi není, však věcí názoru, na čí straně je pravda. Však pochyby o válce nikdo nemaje. I změny na hradě pražském se dějí, my souditi později budeme, zda k dobrému vše bylo. Však jen Všemohoucí rozsudek bude vynášet, a ne prostý lid. Tak svou mysl obraťte směrem jiným.

\lettrine[lines=2,slope=0pt,findent=0pt,nindent=0pt,lraise=-0.2,loversize=0.5]{\capitalInit D}{oba} půstu a odříkání je zde. Však nejen strádáním člověk své tělo tužiti má. Raději ku pozoru se míti, co z mých úst vychází a kterés myšlenky má mysl ukrývá. Pak bojovat o zdravého ducha, muž či žena musí. Zvláště ten, kdo k rytířství tíhne. Rytíř má vzorem býti. Tužit tělo a mysl svou, chránit slabé, věrnosti dostát a vždy prahnout po pravdě, ač to nelehké vždy jest.

\lettrine[lines=2,slope=0pt,findent=0pt,nindent=0pt,lraise=-0.2,loversize=0.5]{\capitalInit V}{e} znamení velký změn rytířské ležení bude. Však nelekej se nikdo. Ležení ve dnech srpnových znovu rozbijeme a očekávati budeme každéhos, kdo pod Korouhví Kristovou se chce bít. Bílého odění jest letos zapotřebí, každý, jinoch či panna musí ho míti. Včasně švadlenu nechte volat. Vzor střihnu vhodného v listech dalších naleznete. Pak i posla rychlého vypravte by v tvrz Litovel dorazil a stvrdil, žes řady naše se zaplní. Vskutku vděčen budu každému, kdos o rytířském ležení se zmíní a my tak v plné síle výpravu započneme. Královská Rada své síly napíná, by vše včas připraveno bylo. I vy svůj oděv vyspravte, meče i štítu bude zapotřebí více než kdy předtím. Kdos praporec svého domu chce nést, bráněno mu nebude.


\lettrine[lines=2,slope=0pt,findent=0pt,nindent=0pt,lraise=-0.2,loversize=0.5]{\capitalInit V}{ám} ostatním kéž ujištění dá Nejvyšší, ten, jenž dává slunci svítit na obloze, jak svým věrným, tak lidu neznabožnému. Potřebujete-li ku rozhodnutí svému zvěděti cokoli, užijte všech vymožeností doby, u mě vždy odpovědi dostanetež. Do té doby Pána našeho a Prvního mezi rytíři všemi, Ježíše Krista, vzývati budu, by v dobrém utvrzoval a od zlého chránil všeckny vás!


\begin{flushright}
{\vspace{-8pt}\fontsize{24}{16}\selectfont Daniel}\\\vspace{6pt}
\normalsize
dáno v~tvrzi Litovel \\
v~den svatého Nikefora L. P. \romanYear
\end{flushright}
\vfill
\begin{center}
  \psvectorian[width=.8\textwidth]{88}
\end{center}
\end{document}